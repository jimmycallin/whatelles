%
% File acl2015.tex
%
% Contact: car@ir.hit.edu.cn, gdzhou@suda.edu.cn
%%
%% Based on the style files for ACL-2014, which were, in turn,
%% Based on the style files for ACL-2013, which were, in turn,
%% Based on the style files for ACL-2012, which were, in turn,
%% based on the style files for ACL-2011, which were, in turn,
%% based on the style files for ACL-2010, which were, in turn,
%% based on the style files for ACL-IJCNLP-2009, which were, in turn,
%% based on the style files for EACL-2009 and IJCNLP-2008...

%% Based on the style files for EACL 2006 by
%%e.agirre@ehu.es or Sergi.Balari@uab.es
%% and that of ACL 08 by Joakim Nivre and Noah Smith

\documentclass[11pt]{article}
\usepackage{acl2015}
\usepackage{times}
\usepackage{url}
\usepackage{latexsym}% -*- program: Value -*-

%\setlength\titlebox{5cm}

% You can expand the titlebox if you need extra space
% to show all the authors. Please do not make the titlebox
% smaller than 5cm (the original size); we will check this
% in the camera-ready version and ask you to change it back.


\title{Cross-Lingual Pronoun Prediction with Feed-Forward Neural Networks}

\author{Jimmy Callin \\
  Uppsala University \\
  {\tt jimmy.callin@gmail.com}}

\date{}

\begin{document}
\maketitle
\begin{abstract}
    We did some fun stuff. All is well.
\end{abstract}


\section{Introduction}

When translating between languages, there are several discourse phenomena that are more difficult to translate than others.
For instance, the act of translating pronouns usually requires insight into what is the antecedent of said pronoun, since gender of noun phrases rarely translate well between languages.
While we have started to see a movement towards machine translation models that do not treat sentences in isolation, discourse oriented systems have yet to introduce performance improvements that would motivate ubiquotous usage.

In light of this, there have previously been attempts at treating pronoun translation as a separate task from common sentence translation.
By doing this, a pronoun translation module could potentially be treated as just another part of translation by discourse oriented machine translation systems, or as a post-processing step in systems where translations using larger contexts than sentence-level simply are not supported (such as Moses).
DiscoMT 2015 introduces the shared task of cross-lingual pronoun translation.

What is given is documents in a source language (English, in this case), a translation to a target language (French), as well as word alignments.
The word alignments are automatically produced by GIZA++.
What is missing, though, are third-person pronouns in the target translation.
While English pronoun usage is relatively easy to infer from the immediate context, this is not the case in all languages.
When translating to French, for instance, the translator requires to keep the immediate antecedent in mind for determining the use of male (\emph{elle}) or female (\emph{elles}) third-person pronoun.
This is something you simply cannot infer from context alone.


\section{Data}

Several training resources were provided as a part of this task.
Europarl, IWSLT, NCV9.
Test data is a collection of transcribed TED talks, in total 12 documents containing 2093 sentences with a total of 1105 classification problems.


\section{Method}

While three separate training data collections were available, in the end only transcribed TED talks from the IWSLT collection were used.


\section{Results}


\bibliographystyle{acl}
\bibliography{acl2015}

\end{document}
